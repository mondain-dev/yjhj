\chapter[聖路善工]{耶穌受難聖路善工}
\section*{聖路善工小引}
紀念天主耶穌受苦受難、是教友成聖之善法。能使人離開罪惡、勉隨德行之道路。因我人類從魔鬼手中救出、由第二位聖子甘受苦難、被釘十字架死而來。所以人想此苦難、及此苦難緣由、爲己之罪。又想至美好之天主、爲至卑賤之罪人、受此大苦、自不得不深恨罪過、願立功勞、以報答其洪恩也。爲此聖教會、常提醒人紀念耶穌苦難。並用多法、引人紀念。

溯自耶穌受難在日路撒冷以後、其地即名聖府、常有人往拜其遺跡。歷代教宗、爲勸勉人到聖府恭敬耶穌苦難、准給許多恩赦。但因普天下之人不能盡到聖府、得此恩典。教宗乃准於通都大邑、建造仿佛日路撒冷聖府者多處。准教眾朝拜、亦得恩赦、一如親到聖府者然。後因建造之處雖多、尚不能盡滿眾人之意。又擴充恩典、使遠方教友、盡能沾此洪恩。其法、即准造十四座十字架、以擬在日路撒冷聖府、耶穌受難之十四處。格肋孟德第十二重准從前教宗所頒恩赦、並諭凡拜此苦路、默想耶穌後難之事、而果遵聖教會規矩、真心痛悔定改者、能得恩赦、亦如親到聖府者然。本篤第十四所准亦如此。並定多少規矩、以善行此神工。又勸各處司鐸、在各本堂內舉行之、引人得此恩典。蓋行此聖路善工、於教眾大有裨益。不但爲得歷代教宗所准恩赦、更能使存想耶穌在十字架上受苦受難之事。而心中感動、或改過自新、或保存義德。總而言之、此乃最中天主聖意之神工也。

歷代教宗究頒幾多恩赦、因年久失傳、無從查考。庇護第十一爲使此事確實清楚、用其至上權能、將以前教宗所頒全限大赦、一律取消、復定新律如下。凡教友靈魂上無大罪、按教宗意求主、默想耶穌苦難。私自或公共按規熱心拜苦路者、每次得一全赦、或爲自己、或爲煉靈皆可。如月內拜苦路十次、而領聖體者、或當日曾領聖體者、則拜苦路時可得另一全赦。凡教友拜苦路時、因故中斷、不能繼續再拜者、則其所拜苦路、每處可得十年限大赦。

爲得苦路大赦、並不規定告解領主、惟須靈魂上無大罪。故於拜苦路前、最好發一上等痛悔。至於舉行聖路善工之法、該有十四座十字架。由有權之神父、得本區主教允准、而祝聖安置之。安置之處、或在堂內、或在公所、凡潔淨幽靜者都可。第一座及第十四座兩十字架、都該放在祭臺或經臺兩旁。其餘十二座按次排列。隨地方寬窄、爲相隔之遠近、略表吾主、從比辣多衙門至墳墓、多經路耳。朝拜時、該當按次一一跪拜、念所當誦之經。倘人眾、一人領經、眾人應之。若只一人、自起自應可也。如有不識字者、只按次默想各十字架處耶穌所受之苦難而念天主經聖母經、各一遍。拜畢、又念天主經、聖母經、聖三光榮頌、各五遍、紀念耶穌五傷。

又念天主經、聖母經、聖三光榮頌、各一遍、按教宗意、求爲聖教廣揚、異端消滅。
凡遇罷工瞻禮之日、並嚴齋內瞻禮六日、爲紀念耶穌苦難至大之洪恩、均當念苦路經。其餘別時、隨自己本意、能彀誠心誦念、爲益莫大焉。

\section*{朝拜念經默想之規}
\begin{enumerate}
    \item[]{\small 祭臺前作好聖號經念以下悔罪經文}
\end{enumerate}
吾主耶穌基利斯督、是真天主、亦真人、造世救世贖世者。我到爾臺前、如久離本家之敗子、又如失路無牧之羊。心中愧悔、莫可名言。今恃爾之仁慈、悔恨一生罪過。不但因失天堂之福、得地獄之刑。實因得罪爾無窮美善可愛之大父、造成救贖我之恩主。伏求因爾所受苦難。赦我往罪、依靠爾聖血功勞、真心定改。從今以後、寧死再不敢犯罪。{\cspace}亞孟。

{\small 奉獻神工} 天主、我等在爾臺前、想爾爲人受難無限之愛情。求爾將此神工、相合於爾無窮聖血功勞。教宗所准恩赦、我都願得。或爲煉獄中親友恩人、隨爾之意。因此懇爾、爲爾之聖教廣揚、異端消滅、及凡教宗所定者。

{\small 首唱者叩拜念} 天主、爲爾所受之苦難、矜憐我等。 {\small 眾應亦然。走行往第一處}

\section*{第一處\quad{\rm\small 在第一座十字架前跪下、後倣此。}}
{\small 首唱者叩拜念} 耶穌基利斯督、我等欽崇爾、讚美爾。

{\small 眾應} 爾因此聖架救贖普世。 \quad{\small 後各處倣此}

此第一處。發顯吾主耶穌在比辣多衙門、受鞭子刺冠厲害苦難之後。皮肉寒冷、血脈流空、諸骨盡露。比辣多聽惡人言語、判斷耶穌該死。\cspace\textbf{默想}我的靈魂、你想一想。那時吾主耶穌、雖然受鞭子刺冠厲害的苦、何等樣安心忍受、聽比辣多定該死之刑。這都爲聽天主聖父、打發祂、爲救贖人的命令。你看這樣順命、痛悔你從前背主命之罪惡、對耶穌云。\cspace\textbf{祈求}吾主、我之天主。爾爲救我、受無數淩辱苦難、及惡官之判斷。因爾寶死、救我永遠之死。{\cspace}亞孟。{\small 首唱者念} 在天我等父者{\small 云云}、{\small 眾應} 我等望爾{\small 云云}。{\small 首唱者念} 萬福瑪利亞{\small 云云}。{\small 眾應} 天主聖母瑪利亞{\small 云云}。 {\small 首唱者叩拜念} 天主、爲爾所受之苦難、矜憐我等。 {\small 眾應亦然。}

{\small 若從第一座十字架、到第二座十字架、相隔幾步、走時亦該念此二句、不拘多少遍、下諸處倣此。}

\section*{第二處}
此第二處。發顯吾主耶穌聽了比辣多審判該死、惡人強耶穌仍穿本衣服、使人認得。又拿一重大木十字架、令之肩負。\cspace\textbf{默想}我的靈魂、你想一想。吾主耶穌肉身疼痛流血、無人憐惜。受惡人許多淩辱、把十字架放在身上、何等樣喜歡背負、走此苦路、爲救你。你今跟隨耶穌、向耶穌云。\cspace\textbf{祈求}我可愛救贖之主、雖爾之仇人將十字架放爾身上、爾甘心爲贖我罪、總不推辭。我自今以後、何敢推辭爾所賞賜、爲聽爾命之十字架、求爾壓服我之偏情、爲能隨爾之意、補贖我罪、承行聖命。{\cspace}亞孟。

\section*{第三處}
此第三處。發顯吾主耶穌背粗重十字架、肉身力盡筋疲、惡人腳踢手拉、故此跌倒在地。\cspace\textbf{默想}我的靈魂、你想一想。天神的主宰、天地的君王、跌倒在人腳下、雖被惡人踐踏、及諸淩辱、總不出一言。此乃教訓你、當如何忍受他人淩辱及諸病苦、無惱無怨。\cspace\textbf{祈求}我之天主、我因爾之壓跌、認我罪很大很重。我今痛悔求爾、以後在患難時、賞我效法爾之忍耐良善。{\cspace}亞孟。

\section*{第四處}
此第四處。發顯吾主耶穌背十字架、路上遇見聖母瑪利亞。\cspace\textbf{默想}我的靈魂、你想一想。吾主耶穌見其母親及聖母見可愛之子、彼此心中如何疼痛、如何憂苦。及彼此心中共相憐惜。聖母見可愛之子卑賤淩辱、滿身流血。吾主耶穌明認母親心內悲傷。此時景況實在可憐、你今對聖母云。\cspace\textbf{祈求}聖母瑪利亞、我之慈母。我是爾子耶穌受苦之緣由、本不敢得爾之矜憐。但爾無窮仁慈、爲我求爾之耶穌、饒恕重罪。及在臨終時、遇見耶穌、領我至天堂永福之所。{\cspace}亞孟。

\section*{第五處}
此第五處。發顯惡人怕吾主耶穌傷重就死、不能到加爾瓦略山受釘、雇一外方人、名呌西滿、幫助背十字架。\cspace\textbf{默想}我的靈魂、你想一想。耶穌如今對你說、西滿所背的十字架、我願你一生常背、蓋耶穌命你忍受祂所賞於你的苦、爲補前罪、是即你的十字架也。如聖經上耶穌說、誰願跟隨我、該背自己的十字架。就是該忍受天主所賞的患難貧病、及各人本分當盡之苦勞。\cspace\textbf{祈求}天主耶穌、我從今以後、定要聽爾命、跟隨爾背十字架、忍受爾所賞之世苦。爲補贖我罪、及躲避永遠地獄之苦。{\cspace}亞孟。

\section*{第六處}
此第六處。發顯有一勇敢聖婦、見耶穌臉上血汗唾污、就分開惡黨、到耶穌跟前、用白帕擦其聖面、聖容就印在帕上。\cspace\textbf{默想}我的靈魂、你想一想。此聖婦的勇敢、雖惡黨眾多、不能阻她行此熱愛耶穌之事。你定真實志向、憑無數三仇內外誘感、當勇敢隨德行的道路、爲報答耶穌、爲你受難的恩典。\cspace\textbf{祈求}吾主耶穌、因爾無窮愛情、受難時留此印像。求爾重新聖容印爾聖容於我心、永遠不滅。效法此聖婦之勇敢、爲能隨爾、及去諸阻我恭敬爾之事。{\cspace}亞孟。

\section*{第七處}
此第七處。發顯吾主耶穌出城時力乏、第二次跌倒。聖傷重開、聖血重流。\cspace\textbf{默想}我的靈魂、你想一想。你的大父、跌在地下、受惡黨壓服、眾人恥笑。你把耶穌的謙遜、與你的驕傲彼此比較、耶穌是天主、聖父的真子、今在眾人以下、卑賤至極。你原來是土、犯罪更爲卑賤、反願在眾人以上。你今速悔謙卑、向吾主耶穌云。\cspace\textbf{祈求}吾天主、我可愛之耶穌、爾爲消滅我之驕傲、跌倒在眾人足下、受無數輕慢淩辱。求爾賦我心中真謙之志向。不斷爲補贖我罪、認己無能、並爲效法爾之心謙、能得靈魂平安。{\cspace}亞孟。

\section*{第八處}
此第八處。發顯許多婦女跟隨吾主耶穌、出城外流淚慟哭。耶穌回頭驚醒、乃說、你們不要哭我的苦難、當哭你們的罪惡、因此罪惡、實爲我受苦之緣由。\cspace\textbf{默想}我的靈魂、你想一想。耶穌很大愛情、雖在將死之時、不斷教訓人、當如何傷痛流淚、救己靈魂。該效法當日那些婦女、並聽耶穌教訓、哭你沒良心、憐愛耶穌、向耶穌云。\cspace\textbf{祈求}吾天主、爾既訓我如何痛哭。爾賞我聖寵、一生不斷傷痛流淚、哭從前罪惡、實爲爾受苦之緣由。{\cspace}亞孟。

\section*{第九處}
此第九處。發顯吾主耶穌到加爾瓦略山下、肉身力量甚乏、第三次跌倒、聖傷全裂、聖口振開。\cspace\textbf{默想}我的靈魂、難道你見耶穌第三次跌倒、還不動心麼。如德亞國人切願耶穌上山被釘、恐祂死在半途、催祂快走。你若常常犯罪、是學惡人的狠心。不斷難爲耶穌。如今快快離開犯罪的機會、向耶穌云。\cspace\textbf{祈求}吾主耶穌、爾既爲我跌倒、受很大窘難。爾可憐我、赦我從前所犯之罪、賜我聖寵、再不敢難爲爾、釘爾在十字架上。{\cspace}亞孟。

\section*{第十處}
此第十處。發顯吾主耶穌到加爾瓦略山上、惡人強剝去衣服、連皮肉都帶去、聖傷重開、聖血重流。惡黨又把酸醋苦膽、強耶穌嗑、加增其苦。\cspace\textbf{默想}我的靈魂、你想一想。吾主耶穌爲潔淨之源、在眾人前、身無衣服、如何羞愧。及受重苦以後、又嗑酸苦的東西、如何受得。這些都爲救你、爲補贖你喜穿華衣、喜食美味、諸快樂肉情之罪惡。你今定改毛病、向耶穌云。\cspace\textbf{祈求}吾主、吾天主。爾既受此淩辱、爲補我罪。求賜我聖寵、克制肉情、並躲避各樣世上虛假光榮。{\cspace}亞孟。

\section*{第十一處}
此第十一處。發顯惡黨放吾主耶穌在十字架上、用大鐵釘穿透祂的手足。\cspace\textbf{默想}我的靈魂、你想一想。你的救贖恩主、釘在十字架、手足如何疼痛。又想十字架苦刑、是如德亞國那時極凶極重之刑罰。吾主耶穌本來無罪、受此苦架淩辱、爲消滅你肉身上私欲偏情的罪惡。你速定改諸罪、把你私欲偏情釘在十字架上、向天主云。\cspace\textbf{祈求}全能天主聖父、爾子耶穌已爲我罪人釘死。求爾看其苦難、饒恕我往日之罪。及賞我勇敢、爲能壓服肉身之偏私。{\cspace}亞孟。

\section*{第十二處}
此第十二處。發顯吾主耶穌聖身懸在十字架上。惡黨搖動、聖傷重開、重流聖血。豎立十字架在土中、兩旁又釘二盜、惡眾大聲譏笑咒駡。其時太陽失光、地動山開、石相觸碎、墳墓自辟。堂中帳幔從上而下分開。\cspace\textbf{默想}我的靈魂、你想一想。其時加爾瓦略山是何景像。天地萬物真主、釘在十字架上。諸無靈的物、認造成之主受難、顯其憂苦。但受難的緣故、非關萬物、實因你的罪惡、豈無靈之物、尚覺憂苦。你有靈的心、竟不動流一滴淚、爲哭耶穌的苦難、及痛恨給此苦難的緣由麼。當真心痛悔、向被釘耶穌云。\cspace\textbf{祈求}吾主耶穌、我之天主、我之凶惡罪過、即爾所受苦難之緣由。我今明認此大關係、真悔所犯罪惡。懇求因爾所流之聖血、洗我靈魂、赦我諸罪。我真心定、永遠不敢得罪吾主。{\cspace}亞孟。

\section*{第十三處}
此第十三處。發顯吾主耶穌死後、有人卸下聖屍、聖母接抱在懷中。\cspace\textbf{默想}我的靈魂、你想一想。聖母瑪利亞、見可愛之子已死、抱在懷中、如何痛苦。諸位聖女及聖若望宗徒見恩主死後淒涼、如何悲傷。你效法聖母及諸聖的痛苦悲傷、哭耶穌的死、並虔心向聖母及諸聖云。\cspace\textbf{祈求}聖母我慈母、我今斷絕從前釘耶穌之罪、爲減少爾之痛苦。定真志向、以後寧死再不敢得罪爾子耶穌。諸聖人聖女、俱爲我轉求吾主耶穌、赦我罪、加我神力、使我永不復犯。{\cspace}亞孟。

\section*{第十四處}
此第十四處。發顯聖母同聖人聖女等、送耶穌埋葬墳墓、蓋以石板。\cspace\textbf{默想}我的靈魂、你想一想。聖母瑪利亞及諸聖用大石掩墳、不見耶穌、心中如何慘傷、如何憂悶。你想到其間、心若不動、明顯爾沒良心、及無痛悔往罪的憑據、倘爾心如鐵石、毫不動情。今把叩拜吾主耶穌的墳墓、思念爲你所受之諸苦難、向耶穌云。\cspace\textbf{祈求}吾主耶穌、我今朝拜爾救贖之墳墓、明見爾這樣愛人至極。我今傷心憂悶、痛哭流淚、暨聖母瑪利亞、及諸聖人聖女悲傷。懇求爾、因此埋葬之功勞、賞賜我消滅於諸罪、晝夜紀念爾之苦難於我心。{\cspace}亞孟。

\hangindent=2zh{\small 首唱者叩拜念} 天主、爲爾所受之苦難、矜憐我等。 {\small 眾應亦然。到祭臺前、首唱者念} 耶穌基利斯督、我等欽崇爾、讚美爾。  {\small 眾應} 爾因此聖架{\small 云云}。

\hangindent=2zh{\small 首唱者念} 耶穌園中祈禱、大發憂悶、汗血遍地。

\hangindent=2zh{\small 眾應} 耶穌基利斯督、我等欽崇爾、讚美爾。{\small 後倣此。}\\
耶穌被茹達斯負賣、惡黨如捕賊然。\\
耶穌被妄證多罪、教首審問如罪人。\\
耶穌被唾面打嘴、掩障聖目、訕毀聖名。\\
耶穌被惡人棄絕、過於大罪人把拉把。\\
耶穌被綁縛於石柱、受鞭打苦刑。\\
耶穌被刺箍聖首、棘刺深入於腦。\\
耶穌無罪而問死罪、親背十字架登山。\\
耶穌負十字時、路上力乏、幾次跌倒。\\
耶穌手足被釘於十字架、在兩盜中。\\
耶穌仰求聖父赦仇、恩賜右盜升天。\\
耶穌在架上語徒若望、表聖母爲我等慈母。\\
耶穌將終曰渴、惡人飲以酸醋苦膽。\\
耶穌死後肋膀被刺、血水盡流。

\hangindent=2zh{\small 首唱者念} 我等欽崇讚美救贖者天主、可愛的大父。天地萬物俱爲我欽崇讚美之。因其受無窮苦難、捨己生命、爲救贖犯命的人類。我等尋思如是恩惠、當狠羞愧己之沒良心。真心悔恨前罪、向耶穌云。 天主耶穌基利斯督、天地萬物大主、人類之救贖。我在爾臺前、認我之錯、告我之罪。及明知所有過惡、真該萬死。但我恩主、今依靠爾、爲我所受之苦難、所流之聖血、真切悔恨前罪。在天地萬物之前、立心定志、一生悔恨補贖、寧死再不敢犯。懇求看爾無窮救贖功勞、饒恕我、賞我聖寵、以後隨爾、全守爾之命令。{\cspace}亞孟。 {\small 默存片刻。}

\section*{耶穌苦難經}
吾主耶穌、在十字架上、全身處處受傷。手足有鐵釘、頭上有刺冠、口嘗醋膽、膚被鞭笞。且明知自己、受此無數艱難、尚與許多人無用、內心憂悶之苦、更甚於外刑。天主爲我如此、我爲天主、不肯離世俗一些之樂、不肯受世俗一毫之苦、有此理否。耶穌受苦、爲補贖我罪、我明知之。今我不從其命、犯罪時豈不輕慢此大恩麼、豈不更加耶穌憂悶麼。

不論什麼人、耶穌甘受其殘傷。宗徒背之、如德亞人告之、外教人害之詈之、司教國主、官員兵民人等、齊心共殺之。因耶穌受難、爲救各種各類之人、所以受各種各類人之苦、我們一總也在其內、若或無信德、或時時犯罪、或無愛慕耶穌之心、還使得麼。

此時聖母何在、在十字架旁哭泣、看耶穌受苦、並不是世俗母子之情、蓋合自己愛人之心、同耶穌之愛、一起獻大祭於天主聖父、以救世人。故聖母爲救世之母、我們個個人、都該學聖母、仰瞻於十字架旁、獻其無盡之功、爲自己犯的罪過、爲別人所犯的罪過、求開恩、求動心、求不負耶穌之苦功。

\begin{enumerate}
    \item[]{\small 念天主經、聖母經、聖三光榮頌、各五遍、爲恭敬耶穌五傷。\\
    又念天主經、聖母經、聖三光榮頌、各一遍、求爲聖教廣揚、異端消滅。作聖號、工畢。}
\end{enumerate}