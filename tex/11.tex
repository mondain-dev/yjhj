\chapter[玫瑰經]{聖母玫瑰經十五端}
\begin{enumerate}
    \item[]{\small 謹按十五端、系聖母瑪利亞一生之事蹟。即吾主耶穌降生、救贖、受難、復活、升天之大端。\\
    吾人習誦正經、靜心默想、大獲神益。\\
    首一分、{\cspace}包含聖母瑪利亞歡喜五端。\\
    中一分、{\cspace}包含聖母瑪利亞痛苦五端。\\
    後一分、{\cspace}包含聖母瑪利亞榮福五端。\\
    每日、或十五端全誦、或分三次誦。若分三日誦、則首一分、宜誦於瞻禮二日、五日。中一分、宜誦於瞻禮三日、六日。後一分、宜誦於瞻禮四日、七日。主日全誦。}
\end{enumerate}

\section*{首一分歡喜}
\subsection*{歡喜一端\quad{\small 天神朝拜童貞瑪利亞、報曰天主特選爲母}}
\begin{enumerate}
    \item[]{\small 誦天主經一遍、聖母經十遍、聖三光榮頌一遍。後十四端倣此。}
\end{enumerate}
獻{\cspace}盛德崇福童貞瑪利亞。我獻此經、敬祝爾聖寵無涯之喜。昔日天神嘉俾厄爾、奉天主之命、恭報於爾云、萬福、瑪利亞。滿被聖寵者、主與爾偕焉。又云、天主聖子選爾爲母、將降孕於爾最淨最純之聖胎、爲人救世。爾於是時、俯躬謙言。我乃主之婢女、願賜成於我、如汝之言。

求{\cspace}今我虔祈聖母、轉祈聖子耶穌、賜我謙遜之德。使我諸凡行爲、自能順承天主至聖之旨。亞孟。

\subsection*{歡喜二端\quad{\small 聖母往見聖婦依撒伯爾}}
獻{\cspace}全備諸德童貞瑪利亞。我獻此經、敬祝爾聖性仁愛之喜。天神嘉俾厄爾、又以爾表姐依撒伯爾受娠之奇事、奉報於爾。爾乃大發熱愛之情、速行往顧。既至其家、伊受孕期及六月。天神報云、宜名若翰。是時若翰在胎、即知爾在伊母之前。又知爾已懷胎、是降生救世之天主。其在胎中、不勝欣躍。聖子耶穌在爾胎中、即赦其所負原祖之罪。依撒伯爾荷天主默照、識爾爲天主之母、不勝歡呼稱讚。女中爾爲殊福、爾胎中子尤爲殊福。我有何德、而煩吾主之母、遠來顧我。

求{\cspace}今我虔祈聖母、轉祈聖子耶穌、賜我愛人之熱心。又賜我凡思言行、罪過之赦、聖恩之錫。及明達天主事理、超性之識。{\cspace}亞孟。

\subsection*{歡喜三端\quad{\small 吾主耶穌基利斯督降誕}}
獻{\cspace}天主聖母童貞瑪利亞。我獻此經、敬祝爾神聖共慶之喜。爾至聖靈魂、所受爾子耶穌誕生於爾至淨至貞之胎、以救世人。爾至喜至敬、裹以裳衣、置於馬槽、俯身拜爲真天主。此時羣天神、奏樂於空中、讚美天主、慶賀世人曰。天主受享榮福於天、良人受享太平於地。

求{\cspace}今我虔祈聖母、轉祈聖子耶穌、賜我甘貧之德。使我輕脫世緣、乃得純心、奉事吾主。亞孟。

\subsection*{歡喜四端\quad{\small 聖母獻耶穌於主堂}}
獻{\cspace}至貞至潔聖母瑪利亞。我獻此經、敬祝爾聖善令譽之喜。聖子耶穌、當時天神顯揚、牧童即來致敬、三王即來朝禮。聖誕後四十日、爾恭抱前往聖殿、獻於天主聖父。此時有一盛德年高西默盎。又有一盛德節婦亞納、讚揚聖子耶穌、真是救世之主。

求{\cspace}今我虔祈聖母、轉祈聖子耶穌、賜我或在聖殿、或居別所、時時皆能普揚天主聖名、讚美天主聖容、令諸人、咸知信從。{\cspace}亞孟。

\subsection*{歡喜五端\quad{\small 耶穌十二齡講道}}
獻{\cspace}勤敏神工聖母瑪利亞。我獻此經、敬祝爾暫憂旋慰之喜。聖子耶穌方一十二齡、隨爾前往聖殿。歸時、相失耶穌。三日夜、爾心痛苦。三日之後、覓至殿中。及見耶穌上座、與耆年博學之士、講論天主事理。一見、不勝欣喜。耶穌同爾言歸、孝敬事爾、至三十年。

求{\cspace}今我虔祈聖母、轉祈聖子耶穌、於我患難之際、賜我神慰。使我時時事事、翕合聖意、克謙克孝、誠事天主。{\cspace}亞孟。

\section*{中一分痛苦}
\subsection{痛苦一端\quad{\small 耶穌山園祈禱}}
獻{\cspace}高上仁慈之母瑪利亞。我獻此經、默思母心痛苦。聖子耶穌三十三歲、受難之期已至。欲救贖人罪、於前一夕、往山園、三次祈求天主聖父。苦懇至極、通體出流血汗。叩拜天主聖父、乞赦人罪。至夜深時、惡眾捕縛、送於亞納斯。

求{\cspace}今我虔祈聖母、轉祈吾主耶穌、賜我能求能禱。又賜我應受苦難合天主旨。並能以我真忍、堪承苦難。{\cspace}亞孟。

\subsection{痛苦二端\quad{\small 耶穌繫受鞭笞}}
獻{\cspace}毅然堅忍之母瑪利亞。我獻此經、默思母心痛苦。聖子耶穌令儀令容、超絕萬眾。今在比辣多衙內、盡褫其衣、繫之石柱、鞭責五千四百有奇。全體剝傷、血流不止。苦痛如是、耶穌默不置辨、有如羔羊。

求{\cspace}今我虔祈聖母、轉祈吾主耶穌、卻卸我世間私欲之衣。賜我真忍、甘受諸艱諸勞、凡天主所賜於我者。{\cspace}亞孟。

\subsection{痛苦三端\quad{\small 耶穌受茨冠之苦辱}}
獻{\cspace}苦刃刺心之母瑪利亞。我獻此經、默思母心痛苦。耶穌我等主、被惡人以棘茨冠、箍於聖額、聖血通流。又以絳色蔽袍披身、偽拜如王、如此侮辱之甚。

求{\cspace}今我虔祈聖母、轉祈吾主耶穌、賜我能悉去自滿驕傲之念、並辭一切虛偽之喜。願爲吾主耶穌基利斯督、忍受患難淩虐之刺。庶望身後、可獲榮福之冠、於天上國至於無窮。{\cspace}亞孟。

\subsection{痛苦四端\quad{\small 耶穌負十字架陟山受死}}
獻{\cspace}寬慰憂患之母瑪利亞。我獻此經、默思母心痛苦。爾極愛之子耶穌、惡黨造一重大十字架、逼令自己肩荷、赴受死之地。負十字架時、一路壓跌難堪。爾隨之慟悼哀傷、泣涕無已。

求{\cspace}今我虔祈聖母、轉祈吾主耶穌、賜我心中、恒覺如是大痛、憶念不忘、亦如身負重大十字架無異。又賜我能勤荷聖孝之十字。{\cspace}亞孟。

\subsection{痛苦五端\quad{\small 耶穌被釘十字架上死}}
獻{\cspace}爲義致命之母瑪利亞。我獻此經、默思母心痛苦。耶穌至於受死之地、被人褫衣、釘手足於十字架上。日月失光、口渴與以醋膽。終命時、天昏地震、石相觸碎、若不堪痛。人人拊胸哀悲、萬物慘傷、皆證被難者、爲造物之真主。

求{\cspace}今我虔祈聖母、轉祈吾主耶穌、藉彼處、忍受痛苦鴻恩、賜我心中能覺悟當時受難之極苦。使我真悔極痛、迅改我一生之罪過。{\cspace}亞孟。

\section*{後一分榮福}
\subsection{榮福一端\quad{\small 耶穌復活}}
獻{\cspace}心慰神怡之母瑪利亞。我獻此經、頌爾安和之榮福。爾極愛之子耶穌、死後第三日復活、身體極光極美。先往見爾、以解爾憂。將爾前日諸痛諸苦、翻作不可勝言之樂。又欲顯厥至愛、屢亦見於宗徒、暨諸弟子。使皆不勝欣躍。

求{\cspace}今我虔祈聖母、轉祈吾主耶穌、賜我善心之真樂、靈魂之潔淨光明、一如復活、不敢再陷於死罪。又使我能輕忽世物、若已死亡、不戀虛妄之福。{\cspace}亞孟。

\subsection{榮福二端\quad{\small 耶穌升天}}
獻{\cspace}神聖共仰之母瑪利亞。我獻此經、頌爾極隆之榮福。爾子耶穌我等主、既已復活。至四十日後、將欲升天之時、面諭宗徒、分行天下。以天主正教、訓誨萬民。與領聖水、洗罪入教。諭畢、自舉升天。古聖羣從、天神簇擁、隨躋天國、坐於全能者天主聖父之右。耶穌命二天神、下慰於爾、及諸聖徒。留爾於世、以啓照保護焉。

求{\cspace}今我虔祈聖母、轉祈吾主耶穌、賜我心能脫離世幻、但愛天上之物。又求爾眷我、顧我、撫護我、行此人世之路、使我畢程、得造天國、永享常生。{\cspace}亞孟。

\subsection{榮福三端\quad{\small 聖神降臨}}
獻{\cspace}上智大能之母瑪利亞。我獻此經、頌爾寵照之榮福、耶穌升天之後十日、爾與宗徒教眾一百二十人、共聚一堂、誦經祈望天主。於辰時聖神降臨、如爾子之所許。迅如風雷、絕無驚恐、賜之撫慰。又有舌形如火、光耀不焚、分置於爾、及眾人之首。於時聖神賜爾以超眾之聖寵、及大聰明。不煩學習、能通萬國方言、傳授聖教。

求{\cspace}今我虔祈聖母、轉祈吾主耶穌、賜我聖寵及發勇力、勉進善德。能以專心、普揚天主聖教。{\cspace}亞孟。

\subsection{榮福四端\quad{\small 聖母榮召升天}}
獻{\cspace}陟天寶座天地元后瑪利亞。我獻此經、頌爾純全之榮福。爾於六十三歲、耶穌遣天神嘉俾阨爾、來報於爾。天主提爾聖身靈魂、並躋天國、受安樂榮福。爾將終時、耶穌顯大聖跡、使向時分行天下之聖徒、咸來辭爾、爾俱一一撫慰。無病無痛、一惟愛慕天主。身終之後、葬爾聖身。天神空中、奏樂三日。以顯爾子耶穌、令爾肉身復活。與爾靈魂、同登天國、享無窮福。

求{\cspace}今我虔祈聖母、轉祈吾主耶穌、賜我臨終時、不陷於邪魔之羅阱。又賜我於此世上、滌惡務善。罪罰已滿、援我升天、見爾聖容、與爾同慶。{\cspace}亞孟。

\subsection{榮福五端\quad{\small 天主立聖母於九品天神之上、以爲天地之母皇、及世人之主保}}
獻{\cspace}巍巍高位之母瑪利亞。我獻此經、頌爾峻德之榮福。爾已升天國、天主聖父、天主聖子、天主聖神、特立爲天地之元后。寵錫榮福、超諸天神、及諸聖人。爲我世人、依賴轉達之主保。

求{\cspace}今我虔祈聖母、轉祈吾主耶穌、憐僕役居此涕泣之谷、賜以聖寵、時垂恩佑。使我死後、得升天國、瞻爾聖容、偕享聖三之永福。{\cspace}亞孟。

\section*{奉事聖母經}
至聖童貞天主聖母瑪利亞。我 {\small 某} 重大罪人、弗敢呼爲主母 {\small \raisebox{-0.2zh}{僕}\raisebox{0.2zh}{婢}} 役、又不敢侍立聖座前。惟恃主母仁慈裕容、乃敢恪恭奉事。特祈護守天神、暨天朝聖人聖女。鑒我愚誠、代我敬懇慈母。爲我恩保、導引我等。自今而後、永遠虔恭、無間無畫。望恩保主母、至聖瑪利亞、爲聖子寶血、許我常與主母所愛之眾、迪我動靜咸宜。又爲轉求吾主耶穌扶右我、凡思言行、永不獲罪。迨我終時、挈我靈魂。得瞻聖子全美聖容、及我主母聖顏。{\cspace}亞孟。

\section*{光明五端}
\begin{enumerate}
    \item[]{\small 二零零二年教宗若望保祿二世在「童貞瑪利亞的玫瑰經」牧函中指出玫瑰經十五端的主題、只包括基督降生、死亡及光榮復活的奧跡、他增添了「光明五端」、是要彰顯基督公開傳教生活的事蹟、突出基督奧跡中光明的時刻──見證生命與福傳的時刻、以滿全玫瑰經的精神、完整地闡明整個基督的奧跡。}
\end{enumerate}

\subsection*{光明一端\quad{\small 耶穌約旦河畔受洗}}
獻{\cspace}救世之母聖母瑪利亞。我獻此經、敬祝爾慷慨奉獻之心。聖子耶穌年屆三十齡、辭爾外出傳教、前往約旦河畔受洗、天主聖父呼爲喜悅之愛子。受洗後由聖神引入曠野、守齋祈禱四十日。受魔誘感、三次擊敗、天神前來伺候在旁。天主子、原無須受洗禮、實爲我人類樹立謙遜之好榜樣。

求{\cspace}今我虔祈聖母、轉祈吾主耶穌、賜我謙卑之心、接受洗禮。誠意悔過自新、一心皈依天主。{\cspace}亞孟。

\subsection*{光明二端\quad{\small 耶穌參與加納婚宴}}
獻{\cspace}仁愛之母聖母瑪利亞。我獻此經、敬祝爾助人爲樂之心。宴席時、爾心細緻、發現酒短缺。聖子臺前爾祈求、愛子允諾變水爲佳釀。眾人面前顯權能、門徒確信其爲主。同聲讚美、光榮天主、伏首叩地感主恩。

求{\cspace}今我虔祈聖母、轉祈吾主耶穌、賜我愛人之誠心、樂於助人、愈顯主榮。{\cspace}亞孟。

\subsection*{光明三端\quad{\small 耶穌宣講天國福音}}
獻{\cspace}善導之母聖母瑪利亞。我獻此經、敬祝爾循循善誘之心。爾子耶穌傳教佈道雲、天國已近、快來悔過自新、接受主福音、活出新誡命。又將真福八端、公佈於眾、建立和平、正義、真理之國。使天主子民生活於新天新地之聖愛中。

求{\cspace}今我虔祈聖母、轉祈吾主耶穌、賜我虛心受教、承行主旨、建設基督神國、永享常生。{\cspace}亞孟。

\subsection*{光明四端\quad{\small 耶穌大博爾山顯聖容}}
獻{\cspace}榮耀之母聖母瑪利亞。我獻此經、敬祝爾享主光榮之心。爾子耶穌偕同愛徒登大博爾山、熱心祈禱、神光煥發、顯現天主聖容。愛徒見之、信心倍增、懇求耶穌常住聖山。天主聖父從天降旨、敦促服從天主聖子。

求{\cspace}今我虔祈聖母、轉祈吾主耶穌、賜我在世遭遇磨難時、仰望基督榮耀、俾能安心忍受暫世痛苦。{\cspace}亞孟。

\subsection*{光明五端\quad{\small 耶穌建立聖體聖事}}
獻{\cspace}聖愛之母聖母瑪利亞。我獻此經、敬祝爾愛人至極之心。爾子耶穌最後晚餐時、明知受難臨近、不願拋棄我等罪人、親立聖體聖事、許諾永留人間。自甘犧牲、作人神糧、養育我等靈魂。又以聖體聖血愛情禮品、伴隨我等奔赴天鄉。

求{\cspace}今我虔祈聖母、轉祈吾主耶穌、賜我常懷感恩之心、敬拜恭領聖體、以愛還愛、在基督逾越奧跡中獲得永生。{\cspace}亞孟。

\begin{enumerate}
    \item[]{\small 念玫瑰經時、聖三光榮頌後加念}
\end{enumerate}

吁!吾耶穌、懇赦我等罪過、免陷地獄永火、爾舉天下人靈、尤其需爾慈悲至亟之靈、左提右挈、悉登天國。

\section*{爲罪人們祈禱奉獻誦}
吁!耶穌、我行此功、敬以奉獻、爲伸我愛爾之情、爲求罪人悔改、爲懇護佑當今教宗:又爲賠補世人罪惡、上慰聖母無玷之心。

\section*{呼籲法蒂瑪聖母誦\quad{\rm\small 每瞻禮七,爲首瞻禮七虔誦此經,每次可得百日大赦}}
吁!法蒂瑪聖母、玫瑰之后、往者承爾降現葡國、賜以和平、即消內憂、並解外患。今我中國、水深火熱、懇迴仁目、憐視憫恤。爾子中國子女、競趨爾前、奉獻敬禮。望爾大展神能、恩賜渴望之和平與堅定之信仰、使道德重振、風氣轉移。耶穌聖心與吾母無玷之心、當能蔭庇中華、平定聖教艱難、使普世萬民、親睦相處、如兄如弟、繞膝歡呼「和平之后。」{\cspace}亞孟。

法蒂瑪聖母、爲我等祈。玫瑰經之后、爲我等祈。在天中國之后、 爲我等祈。 

玫瑰經珠兮、 普世萬民之救星;瑪利亞無玷之心兮、 罪人所仰望而終身。

\begin{quote}\bfseries 求聖若瑟垂佑誦\end{quote}
