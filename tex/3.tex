
\clearpage
\markboth{}{}
% \addcontentsline{toc}{chapter}{彌撒規程}
\chapter[彌撒規程]{}
\section*{與彌撒禮}
{\small 彌撒將行、皆鞠躬跪拜、作聖號、誦定心祝文。鐸德詣臺稍退、輔彌撒者、誦解罪經、眾宜默默同誦、同三拊心、以表痛悔真切。見鐸德轉身、即叩首。鐸德就臺右誦經時、宜立、同作聖號、見鐸德下跪、亦跪而叩首。鐸德盥手畢轉身、眾人皆跪、默思吾主、爲我等重罪受難諸恩。舉揚聖體及聖爵時、俱三拊心、眾誦舉揚聖體聖爵各祝文。鐸德領聖體時、眾當以神領受、共沾耶穌大恩。鐸德畫十字祝福於眾、即叩首領受彌撒將畢、鐸德詣臺右誦經、眾共謝恩。此總撮禮儀大略、已便易曉而已、若其詳、見\BookTitle{彌撒祭義}。}

\begingroup
% \addtocontents{toc}{\cftpagenumberson{chapter}}
\let\clearpage\relax
\chapter*{彌撒規程}
\endgroup

\section*{善願五端}
\noindent 恨我一生自作愆尤、並惡古往今來罪過。我若尚能阻止、祈主賜佑阻之。

\noindent 讚揚開闢至今德業、頌羨今迄世末善工。我若猶能增廣、祈主賜佑增之。

\noindent 願我思言行爲、能具諸聖所具。將具、及克具之善志、爲得愈顯天主光榮。

\noindent 凡我仇讎辱我面、毀我名、肆害我、陰謀我。弗論何故、悉以真心寬恕。

\noindent 若能普救世人升天、爲之逐一捐軀、力本不逮、心實甚甘。祈主賜寵遂願。

\begin{quote}
\bfseries 天主經\quad 聖母經
\end{quote}

{\small 神父著衣時、當念}
\begin{quote}
    \bfseries 信德誦\ 望德誦\ 愛德誦\ 天主經\ 聖母經
\end{quote}

\noindent{\small 神父灑聖水、當念}
\section*{灑聖水經}
望吾主灑我、而我自潔。洗我、而我自白、白於雪。

\begin{quote}
    \bfseries 天主經\quad 聖母經
\end{quote}

\section*{與彌撒規程}
{\small 彌撒禮節、約有三端、曰將祭、曰正祭、曰徹祭。將祭十有八條、正祭九條、徹祭六條。今撮其略於後、以便日用。}

% \pdfbookmark[section]{將祭}{mass1}
\section{將祭}
\begin{enumerate}
    \item[一、] {\small 鐸德詣臺退下者、示謙抑不敢當祭、亦以萃一堂之精神、莫不沖凜也。與彌撒者、此時當致心兢業、思我大罪人、參與大祭、虔恭比鐸德更當何如。〇當念定心祝文。}
\end{enumerate}

\section*{定心祝文}
至慈之大父、共慰之天主、至恕至寬。令惟一聖子、爲救我眾、釘於十字架上。又垂諭以所獻天主最重之禮、每日復行於聖教會、增益我力。吁、惟此禮、高厚淵微、能闡主愛、能保我等於真福。伏惟佑我有事者專心致敬。庶於如是洪恩、定受其賜。爲我等主基利斯督。{\cspace}亞孟。

\begin{enumerate}
    \item[二、]{\small 鐸德畫聖號、稱天主三位一體之名、以表未有天地之先、獨有一主、而一主內即含三位。以其全能全知全善、造成萬類、即以其無窮福德、公之於物、使各享受。與彌撒者、至此當敬心感謝天主、造此世界、令我享用、當何等發心奉事乎。〇當念鐸德畫聖號經}
    \begin{quote}
    \bfseries 聖号經
    \end{quote}    
\end{enumerate}
\begin{enumerate}
    \item[三、]{\small 鐸德悔罪誦悔罪經、指人生各有原罪、又有自造之罪、必大痛悔、方得拔解。與彌撒者、至此當想一生之罪、真切愧悔、定志遷改、求主赦宥、方敢與祭見天主也。隨輔彌撒者默念}
    \begin{quote}
        \bfseries 解罪經
    \end{quote}
\end{enumerate}
 
\begin{enumerate}
    \item[四、]{\small 鐸德上臺、就臺左誦古經一段、指天主未降生前、古聖祈望天主降生赦贖萬民。與彌撒者、當知降生、誠緣不偶、正欲盡開天路、引我等共升、若肯真心悔罪欽崇、必蒙吾主救拔〇當念}
\end{enumerate}
我重罪人、憶昔古聖人、望吾主降生、救贖我等重罪。求天主聖父之全能、賜我等力量、翼我柔懦。求天主聖子之全知、賜我明悟、啓我昏昧。求天主聖神之全善、賜我真切愛慕、振我懈惰。

\begin{enumerate}
    \item[五、]{\small 鐸德就臺中誦經、曰天主矜憐我等、基利斯督矜憐我等、九遍、指古聖痛世罪重、哀求真主、早降臨救世。九遍者、每位哀求三次。與彌撒者、至此當想己罪種種、求赦於聖三。又求聖父全能、賜我力量。求聖子全知、賜我明覺。求聖神全善、賜我真切愛慕之心。〇當念}
\end{enumerate}
吾主、矜憐我等罪人。 {\small 三遍}

\begin{enumerate}
    \item[六、]{\small 鐸德將手兩開、復合掌、誦欽崇榮福、上則榮福於天、下則安和於善人、是吾主降生後、天神在空中所歌讚者。與彌撒者、當陪天神讚頌上主、降我人間、務求自新、不負主恩也。〇當念榮福經}
\end{enumerate}

\section*{榮福經}
天主受享榮福於天、良人受享太和於世。我等稱頌爾、讚美爾。欽崇爾、顯揚爾。爲爾大榮謝爾。主、天主、天上之君、全能天主聖父。主、惟一聖子耶穌基利斯督。主、天主、天主羔羊、聖父之子。除免世罪者、矜憐我等。除免世罪、受我等禱。坐聖父之右、矜憐我等。蓋耶穌基利斯督爾惟一聖、惟一主、惟一至上。偕聖神、於天主聖父之榮福。{\cspace}亞孟。

\begin{enumerate}
    \item[七、]{\small 鐸德轉身對眾云、主與爾偕焉、指吾主降生後、傳播於世、爲人類稱慶者與彌撒者、當感吾主降生大恩、幸沐其教、求主加寵、降在我中、永不相離、又求腁人間、人人歸向。〇當念}
\end{enumerate}
謝主、賜我得知降臨我心中。更求令普天下之人盡知殊恩、罔不歸向。

\begin{enumerate}
    \item[八、]{\small 鐸德就臺左開手誦經、隨本日瞻禮之義祈禱。與彌撒者、至此當申達所願於主前、不拘何等、但非惡事、皆可求也。又要知我所宜得、主必賜我、若不宜得而不與、亦當順命聽主命。求之先、當備一事、事合主之心、求之後、應定一懇懇不倦之念、該我輩求主、關係急切、如何可徒口求、又何可懶求耶。}
\end{enumerate}
\section*{七祈求}
\hangindent=1.5zw 一、求爲聖教宗主。祈主允延德壽、化及萬方。

二、求爲當今元首官長。祈主賜以四方寧靖、五穀豐饒。

三、求爲主教並諸位司鐸。祈主賜以神形兼佑、德化日隆。

\hangindent=1.5zw 四、求爲父母親友恩人。祈主保佑和睦平安、欣勤守誡。及未進教者、棄邪歸正。

五、求爲疾病貧窮患難者。祈主賜以安寧、化殃爲吉。

六、求爲諸異端者。祈主消滅邪妄、咸歸正教。

\hangindent=1.5zw 七、求爲煉處靈魂、及近亡諸信者。懇祈聖母轉求天主、寬煉往罪、早賜升天。

\begin{quote}
\bfseries 天主經\quad 聖母經
\end{quote}

\begin{enumerate}
    \item[九、]{\small }鐸德按手經上、誦經一段、指古聖所紀、論天主降生事、與後來符合。又指若翰保弟斯大略在降生前、令人痛改、以俟主臨。與彌撒者、當思凡有阻礙於主者、預爲清辟、以俟主來。
    \item[十、]{\small 移經臺右、鐸德奉香誦聖經、約有七端禮儀、各有深義。一、請經從臺左至臺右、指吾主降生後、聖教從如德亞國傳於他國也。二、鐸德臺中鞠躬、祈主神佑傳教於他人也。三、眾人站立、聽經語、指勃然振起、必欲遵行也。四、奉香於經三次。五、奉燭二座、蓋尊重其經、指經義之美、猶如奇馥之香、而賜人以靈光之照也。六、畫聖號於經於額、於口、於胸、蓋此經、是受釘十字架之主所命者、以畫十字於經額口胸、不但不可生妄想、妄言、妄行、而實當時念時言時行也。七、念畢親經、指經味無窮、含咀不盡也。} 
    \item[十一、]{\small 鐸德就臺中誦信經、指聖教既傳萬國、人人敬服而信從之。與彌撒者、想我若非奉教人、得於何處用功耶、當跪同鐸德念。}
    \begin{quote}
        \bfseries 信經\quad{\rm\small 至我信其因聖神降孕、跪下。}
    \end{quote}
    \item[十二、]{\small 又轉身念主與爾偕焉、上言萬民信從、則真主在人心內、今鐸德轉身慶慰眾人、言主在爾輩中矣。與彌撒者、當想既奉教後、真主親臨、不爲無據、又恐此後或逆主命、復失其寵、難以稱此語也、醒之醒之。〇當念}
\end{enumerate} 
謝吾主、賜我得知降臨我心中。

\begin{enumerate}
    \item[十三、]{\small 開聖爵奉面餅、又斟葡萄酒於中、以是二品奉獻、指吾主降世贖人、不惟以正言訓人、以聖跡立表、且願以己身爲祭品、奉獻於天主聖父、以息義恕、爵中酌酒在祝聖後、即是吾主聖血、流注於十字架上、爲洗萬罪之寶藥。與彌撒者、至此當想己負重罪、賴主大恩、捨身贖我、我當何如報答。此時當以己身心、俱奉天主、定我一身諸事、悉遵主命、即爲天主致命、亦所不辭、以報主恩於萬萬分之一。鐸德兩手、捧面餅祝誦經言。〇當念}
\end{enumerate} 
吾主耶穌基利斯督、至仁至慈。慨發重寶、贖我等罪。{\small 兩遍}
\section*{欽敬聖體仁愛經}
至慈吾主耶穌基利斯督、自甘受傷傾血、釘死在十字架上、爲救贖眾罪。又永留聖體大祭、養人靈、醫人心、更顯其愛於無窮極。罪人 某 等、感受如是洪恩。痛悔往愆、大發心願。將我神形、盡獻吾主。時時欽敬、思慕聖體。懇賜恩佑改過、堅定我心。令我以毫釐仁愛、報主無窮仁愛。得至善終、享天上仁主永遠真福。{\cspace}亞孟。

\begin{enumerate}
    \item[]{\small 鐸德兩手捧聖爵、祝誦經言。〇當念}
\end{enumerate} 
吾主、以此聖血、救我等罪。倘若爲主致命、所深願焉。 {\small 兩遍}

\begin{enumerate}
    \item[十四、]{\small 鐸德盥手、轉身向眾、請虔心同禱。與彌撒者、至此當想、洗心滌慮、固是本分、而一念虔誠、必使洗之又洗、滌之又滌、以至於無復可洗滌、乃爲敬謹之至。〇當念向耶穌聖靈誦。}
    \begin{quote}\bfseries 小悔罪經\end{quote}

\end{enumerate} 
\begin{enumerate}
    \item[]{\small 鐸德轉身向眾、當念}
\end{enumerate} 

\section*{向耶穌聖靈誦\quad{\rm\small 此聖依納爵、日誦原文。}}
懇祈吾主耶穌基利斯督、聖靈恩寵我。耶穌聖軀扶佑我。耶穌聖血酣暢我。耶穌聖肋之水潔清我。耶穌苦難聖死堅勵我。嗚呼、伏求全善耶穌垂允、藏我於聖傷中。勿棄我、勿許我離背。望救我於諸惡仇。迄我死後、召我趨赴主臺前。偕諸神聖、讚揚吾主、於無窮世。{\cspace}亞孟。
吾主耶穌、爾於斯奇奧之禮、初遺聖愛之表、受難之跡。乞賦我愛爾之誠心。使我如是領爾聖體、可時獲爾救我之洪恩。

\begin{enumerate}
    \item[十五、]{\small 鐸德微聲念經、指吾主未受難前、暫居他方、不顯出於如德亞都城、非爲避難、時未至也。又以訓我等、當亂世出處、亦宜相時而行。}
    \item[十六、]{\small 鐸德高聲、念三多斯云云、指吾主近受難之期、自遠方、進如德亞都城、雖有不知而妬害者、其敬者、亦不啻億萬、扳花迎接、高聲讚賀。所念三多斯云者、譯言聖、聖、聖、正是讚賀之言也。與彌撒者、至此當思吾主、爲我輩受難、屬其夙願樂爲、而我輩事主最容易者、尚不肯爲、愈可羞愧。又以見事情難定、今日尊敬我之人、即五日後、侮慢我之人。正警人不可輕聽浮言、輒自矜喜也。〇當拊心三次、默念、}
\end{enumerate}
我罪、我罪、告我大罪。
\begin{enumerate}
    \item[十七、]{\small 鐸德合掌、默存祈求天主廣揚聖教會、保佑當今教宗、國家元首、本主教、及眾司鐸諸品者、並父母親友恩人、及普天下之奉教者。此指吾主受難前一晚、在園中默禱聖父、奉己所受諸苦、代民贖罪、又想我縱受苦尚有不肯感發遷改、辜負我恩者、作如是想、忽然汗血流地。吾主受苦如此、與彌撒者、至此當存想此意、偕吾主、偕鐸德、祈求天主、大開天下人心、教化盛行、使人人不負吾主耶穌至情。〇當念}
\end{enumerate}
求吾主廣揚聖教會。保佑當今教宗 某、本主教及眾司鐸諸品者。並父母親友恩人、及普天下奉教者。

吾主將受難之前一夕、定聖體大禮於宗徒。在山園中作別、爲我等萬民、默祈聖父、受苦贖罪。尚有不能感發其情、盡解之罪、辜負其苦。作如是想、忽然汗血迸流。內情所發、至慈之恩也。伏祈吾主普化之。

\begin{enumerate}
    \item[]{\small 與彌撒者、至此應祈求、爲生者、死者、病者、患難者。}
\end{enumerate}
 
\section*{求恩祝文\quad{\small 此聖加祿原文、人能常誦、或彌撒中虔誦、俱蒙神效。}}
罪人罪大惡極、無可奉獻聖父。願仰藉吾主耶穌聖體聖血功勞、以獻天主聖父。  罪人罪大惡極、無可奉獻耶穌。願仰藉聖母瑪利亞功行苦勞、以獻吾主耶穌。  罪人罪大惡極、無可奉獻聖神。願仰藉諸聖熱愛衷情、以獻天主聖神。  罪人專懇耶穌聖母、及諸聖人聖女同祈天主。加佑當今教宗 某、及天下諸聖教會、教化衍盛、異端消滅。人民清泰、國土安靖。未奉教者、普賜進教事主。已進教者、普賜遵教立功。在煉獄者、普賜赦罪升天。遭苦難者、普賜平安得所。  罪人過惡深重、不敢瑣覿真主。總仰藉天上諸功、普世諸功、獻主臺前。求主早擴聖教、速救萬民、赦我等罪過、拔我等罪根。除我等偏情、加我等力量。守誡行善、以至死後、救我等靈魂升天、允我等祈求、滿所冀願。{\cspace}亞孟。

\begin{enumerate}
    \item[十八、]{\small 鐸德於聖爵上、先後畫許多十字、此指吾主受難日、諸般苦痛也。與彌撒者、至此當感謝主恩、即求主賜我日日負我十字架、隨從吾主。吾輩每日遇難忍之事、爲主而忍、又所當行克己之功、皆爲我之十字架也。〇當念} 
\end{enumerate}
願吾主至仁之體、普現寵臨。 {\small 二遍}

% \pdfbookmark[section]{正祭}{mass2}
\section{正祭}
\begin{enumerate}
    \item[一、]{\small 舉揚聖體聖爵、指吾主耶穌爲我被釘十字架上、莫大之恩也。與彌撒者、至此思吾主受難、即不靈之物、且表其哀情、我輩人類、可不感發動心哀悔遷改乎。又有舉揚之義、因吾主耶穌甘下受苦辱、將來天主聖父、必與以審判之權、而加諸萬國萬民之上、爲眾瞻仰。與彌撒者、知受難之恩、又知舉揚之義、則悲喜交集、而愛慕不已矣、跪下即叩首。〇當念}
\end{enumerate}
\section*{舉揚聖體祝文}
伏望天主聖父、看吾主耶穌聖體分上、赦我等之罪。
\begin{enumerate}
    \item[]{\small 鐸德舉揚聖體、即鞠躬三拊心。當念}
\end{enumerate}
俯拜稱謝吾主耶穌基利斯督、甘願以此十字聖架、救贖普世。伏祈吾主、赦我等罪。{\cspace}亞孟。
\begin{quote}
    \bfseries 小悔罪經
\end{quote}
\begin{enumerate}
    \item[]{\small 鐸德跪下、叩首、當念}
\end{enumerate}
\section*{舉揚聖爵祝文}
伏望天主聖父、看吾主耶穌聖血分上、沃我等心身。
\begin{enumerate}
    \item[]{\small 鐸德舉揚聖血、即鞠躬三拊心。當念}
\end{enumerate}
申爾福、至寶血、吾主耶穌基利斯督、在十字臺上、爲人永福所留下者。
\section*{朝拜耶穌聖血誦\quad{\rm\small 每日進堂時、朝拜聖血七次、每次念天主經聖母經各一遍、務須自首自尾、次第行之。每次朝拜、宜少加存想、行之不倦、大有神益。}}
我朝割肉聖血。求賜上智之恩、與絕欲之德。

我朝園中流汗聖血。求賜明悟之恩、與節食之德。

我朝石柱上受撻所流聖血。求賜識見之恩、與仁愛之德。

我朝刺箍聖首所流聖血。求賜敬畏之恩、與謙遜之德。

我朝釘手聖血。求賜善謀之恩、與哀矜之德。

我朝釘足聖血。求賜剛毅之恩、與恒久之德。

我朝剽胸聖血。求賜欽崇之恩、與含忍之德。

\begin{enumerate}
    \item[二、]{\small 神鐸德於聖體上、畫十字聖號五次、指吾主受難、遍身皆苦、而手、足、胸旁、五處重傷、尤非他恩可比、故宜時時誦念、較之感頌眾恩、尤爲吃緊。與彌撒者、日日固誦五傷經、以求五德、此時更宜感激、五傷宛然在目、五德莫然銀欣焉。〇當拜誦五傷經。}
\end{enumerate}
\section*{五傷經}
\hangindent=1zw 一、拜右手之傷、求勇德。以輕世福、不致驕傲。祈吾主、救我等心身。{\small 誦天主經、聖母經、各一遍、後同。}

二、拜左手之傷、求忍德。以勝世禍、不致失望。祈吾主、救親友恩人。

三、拜右足之傷、求勤德。以趨諸善、得升天堂。祈吾主、救煉處靈魂。

四、拜左足之傷、求畏德。以避諸惡、免墮地獄。祈吾主、救犯罪惡人。

\hangindent=1zw 五、拜肋旁之傷、求愛德。上愛天主、下愛眾人。祈吾主、救讎仇害我。 {\small 拜畢念}

吾主耶穌、極珍至潔、聖體之五傷。我今伏拜瞻仰、真爲耶穌所出、以贖我值之五寶。耶穌所印、以記我之五號。耶穌所啓、以盼我之五眼。耶穌所掘、以飲我之五泉。耶穌所垂、以援我之五綆。耶穌所備、以升天之五門。所以我今懷念深想、不能不心痛衷熱。感激吾主耶穌、贖我、記我、盼我、飲我、援我升天、若此之莫大恩功。懇求吾主耶穌、不以我辜恩大罪、絕我棄我、干犯聖誡、而自失此極珍至潔五寶、五號、五眼、五泉、五綆、五門、無窮之大益。{\cspace}亞孟。

\begin{enumerate}
    \item[三、]{\small 鐸德又合掌默存、指吾主死後、救古聖靈魂升天。故鐸德至此、爲奉教已歿、在煉獄者祈求。與彌撒者、亦當於此致情、蓋理當相通功也。〇當念}
\end{enumerate}
求吾主、救煉獄眾靈。賴主仁慈、息止安所。{\cspace}亞孟。
\begin{enumerate}
    \item[]{\small 念天主經、聖母經、各一遍。}
    \item[四、]{\small 鐸德拊心、以示吾主憐念眾罪人。與彌撒者、至此當悔恨前罪、以求恩赦。〇鐸德拊心一次、我等拊心三次念。}
\end{enumerate} 
吾主矜憐我等。
\begin{enumerate}
    \item[五、]{\small 鐸德在聖體上畫十字、望彌撒者、亦當畫十字、拊心三次念。}
\end{enumerate} 
吾主、救我等罪人。
\begin{enumerate}
    \item[六、]{\small 鐸德高聲誦天主經、在天我等云云、我等肉身靈魂、生前死後、所望於主者、全在主經之七求。此時聖體在上、正該感吾主代人贖罪、於此不求、更待何時求乎。故鐸德自己立表、高聲朗誦、教人不可錯過此會。與彌撒者、思平日固誦、而此時同鐸德聲聲喚醒、心神融洽於其中、又爲親切。〇當念}
    \begin{quote}
        \bfseries 天主經\quad{\rm\small 一遍、要默想經中意義、如親對主前求說、又念}
    \end{quote}
\end{enumerate} 
吾主、吾肉身靈魂、生前死後、望於天主。悉在主經之七求、伏惟俯允。 {\small 二遍}
\begin{enumerate}
    \item[七、]{\small 面形分開爲三、蓋指吾主在十字架上命終、靈魂與肉軀相離、又復生升天、而加全福於天上人間靈薄三處。又已過現在未來三等人、無不盡被吾主之恩。與彌撒者、思己靈魂與肉身離時、若無功德、安得見吾主而承永福。又思吾主如此大功、三處均庇、敢不在此世、急於戰勝三仇、使古聖先升者、得引手援我、同煉罪者、齊登天國乎。〇當念}
\end{enumerate} 
我情願與肉身分離、不願離吾主。{\small 二遍}
\begin{enumerate}
    \item[八、]{\small 鐸德請小分面形、畫十字、入聖爵、以示主之安和在爾輩、乃鐸德祝願眾人之詞也。亦指吾主復生後、現慰宗徒曰、我之安和在爾輩中也、宗徒甚喜。與彌撒者、至此當自發一甦生之意、絕去日前罪過、心中光明迅捷、安穩無礙、即是吾主之安和在我也。}
    \item[九、]{\small 鐸德叩心痛悔、隨領聖體、聖體系養靈魂之寶藥、必先去罪根、然後用之有益、鐸德雖已悔罪、至此復叩心悔之、恐悔或未盡也。與彌撒者、當參想此意。不領者、發一熱心、切望求領。將領者、發一喜心、恭迎吾主降我心中。然貴實領、貴神領、不可徒領。〇當念神領聖體誦。}
\end{enumerate} 
\section*{神領聖體誦}
吾主、我切望領聖體、使我神魂與聖體相依。誠恐罪愆、不能滌淨、不敢輕領。祈吾主、啓我改過遷善、將來定要求領、沾無極恩寵。專心冀願、若恭領之者然。{\cspace}亞孟。{\small 又念}
救世之天主、救我等。 {\small 三遍又念聖體聖母合讚。}
\section*{聖體聖母合讚}
吾主至聖之體、我等願常爲讚美。又我眾共尊之母、卒世童貞聖母瑪利亞無原罪之始胎、並爲讚美。{\cspace}亞孟。
\section*{求領聖體祝文}
維茲聖體、領之何爲。耶穌聖身聖血聖魂、無不具備。聖父聖子聖神、允咸赫臨。重罪之人、何容妄領。但吾主勸諭、領者必獲天堂常生、又何敢不領。今我神糧匱乏、力量虛微。願滌我心、願堅我德。慎我言動、行我苦工。提正抑邪、心切神領。庶幾鑒此微誠、不嫌愚蒙、特賜實領。臨格我心、常存寵照。{\cspace}亞孟。
\section*{領聖體前誦}
聖人若翰、自胎時爲聖、嘗謙言曰、罔敢與主釋屣帶。聖伯多祿宗徒之長、尚曰、吾主耶穌、當遠我、勿近我、我乃大罪人。二大聖如此謙遜敬畏、我且如何。我罪惡彌深、曷敢近主。主降世時、療痊聾瞽瘡疾諸人、命死者復活、每與罪人言談坐食。我常得罪、猶如聾瞽瘡疾、如死者然。求主療我活我、望賜聖體。但我罪人、何敢妄領。因主昔有遺言曰、人若不領主聖體、罔能生活。乞賜領受。{\cspace}亞孟。
\begin{enumerate}
    \item[十、]{\small 鐸德跪下、起、開聖爵、當念}
    \begin{quote}\bfseries 解罪經\end{quote}
    \item[十一、]{\small 鐸德捧聖體轉身向眾、當念}
\end{enumerate}
\section*{捧聖體時誦}
卑污罪人、辱主俯臨我心、曷以當之。伏惟吾主特降一命、我心諸疾、必痊癒矣。
\begin{enumerate}
    \item[十二、]{\small 鐸德捧聖體送下、當念}
    \begin{quote}\bfseries 小悔罪經\end{quote}
\end{enumerate}
\section*{領聖體時誦\quad{\rm\small 隨領聖體念}}
祈望耶穌靈魂聖我、耶穌聖體救我、耶穌聖血淨我。{\small 領畢則止}
\section*{已領聖體祝文}
我謝天主、我感天主。今日臨我胸中、比死後援我升天、厥恩更深、厥功更大。我今永時永日永歲、常憶天主救贖之奇愛。常思聖母敬供之隆情、常如三王來朝之盛禮。周日存想、一生不變。庶得天主、常存我心、常持我口、常保我身。{\cspace}亞孟。
\section*{領聖體後誦}
至仁至慈天主。我受主無極恩惠、無可稱謝、我重罪多惡、思言行爲、無不得罪、身神污穢。主不但不罰我、更寬裕待我。肫切動我、悔恨前愆、改遷無怠。更忘我罪、容我近主、領主聖體。令我神內、得懷上天下地、無比珍美。主在世時、凡誠心近主、無不取益適願。有罪者被化、改惡遷善。病者獲愈、憂者獲慰、苦者獲安、愚者獲明。今蒙主仁慈、得我罪之赦、我病之愈、我憂之慰、我苦之安、我愚之明。主在我心、爲我心主、求主常居勿棄我。以主聖意、爲我心志、庶恒懷主、須臾不離。善生安死、偕主享主、至於無窮。{\cspace}亞孟。
感謝吾主耶穌、無價珍寶神糧。賜助神力、免陷罪惡、易走天堂之路。 {\small 二遍}

% \pdfbookmark[section]{徹祭}{mass3}
\section{徹祭}
\begin{enumerate}
    \item[一、]{\small 鐸德既斂聖爵等物、又請經復過臺左、誦經一段、即轉身仍念主與爾偕焉、示吾主在爾輩中焉。蓋指吾主之教、從如德亞而傳於萬方、如德亞之民、多反失之、故終必復傳於如德亞。又吾主因救世、而於如德亞降生、終必從如德亞降來大審判。與彌撒者、至此當想未聞教之先、猶可諉於不知、即聞教後、不可不猛省力行、備死後審。}
\end{enumerate}
謝吾主、降臨我等中焉。今日乃吾主允我來瞻仰。我今在此、恐不能正心誠意、反開獲罪之端、祈寬恕之。
\begin{enumerate}
    \item[二、]{\small 鐸德又誦經謝恩。吾主之恩、當時時奉謝、然或有受恩而不知者、至審判日善惡判露、賞罰公嚴、始知吾主當初、所賜之恩深厚也、得不感謝無窮乎。與彌撒者、思彌撒之恩難逢、天主今日、使我得與此祭、至審判日、必爲天主所寵、而投誠誦謝、自不能已矣。}
    \item[三、]{\small 鐸德又轉身、念主與爾偕焉、此鐸德賀眾人之領聖體、及向與彌撒者、言主在爾輩中焉。又指此生前嚴加修省、必使善念刻刻相續、一刻不間、方爲萬德渾全、而與主翕合也。}
    \item[四、]{\small 鐸德又向眾念曰、依此彌撒、允洽各眾之心。願祈求。與彌撒者當轉思彌撒雖畢、而或有忽略錯過、雖與猶不與也、即當退省、日加精進、以得與此大祭爲喜。〇當念}
\end{enumerate}
今日幸承天主不棄、得與諸天神同堂瞻禮。日後審判、冀免地獄之苦、獲遂升天之望。{\cspace}亞孟。
\begin{enumerate}
    \item[五、]{\small 鐸德畫十字祝福於眾、此時鐸德稱三位一體、祝願奉教者、代天主降福於眾人。又指吾主日後、挈善人同升天國享榮福也。與彌撒者、想日後升天之福、斷當勉勵相警、無使空爲善跡、而致失真福也、〇當叩首感謝、畫十字。念}
\end{enumerate}
\section*{降福祝文\quad{\rm\small 煉獄彌撒不念}}
伏惟全能至仁者天主聖父、及聖子、及聖神、降福保全我眾。{\cspace}亞孟。{\small 念畢立起}
\begin{enumerate}
    \item[六、]{\small 鐸德立臺右、誦經一段、捧聖爵歸、禮畢。古禮以右爲尊、左者、天堂真福之處也、指世界盡時、升天得福之人、永遠在天、讚謝天主。與彌撒者、當思世福至小、至暫、至雜、天福至大、至永、至純。與其徇目前暫且離之小樂、何如圖永遠大且純之真福乎。固凡我同類、隨賢愚貴賤、老幼男女、及殘疾疲癃者、一體隨時誘勸、必以吾主爲父、在天爲歸、欽崇爲事、日寫其理、日除其礙、日勉其進。與彌撒時、細繹所行之義、與彌撒後、密存所得之精義、以至日後、得明見吾主、人人升天受福、如此、方不負吾主立彌撒之大義、而我等與彌撒之功亦全矣。〇彌撒畢當念}
\end{enumerate}
吾主耶穌、我專心感謝爾。賜我等與彌撒、見爾於祭品中。沾爾洪恩何樂如之。伏祈爾、俾罪人於此世上、恪守爾命。不敢失望、得升天國。{\cspace}亞孟。
\begin{enumerate}
    \item[]{\small 凡大齋、施捨、克己、功勞有限、惟虔心專情看彌撒奉獻天主、斯受恩無限耳。一、要認天主掌握生死之大權。二、要謝主無窮恩愛。三、要每日獻此功、補贖所犯之罪、求主赦免。四、望主施恩。昔主降生時、牛驢尚知近主、吾人豈可泛泛忽忽哉。況彌撒時、天主天神俱在主臺前、吾等卑污下賤、敢不恭敬、而反褒瀆乎。奉教與彌撒者、口誦心存、庶無辜此大禮、以共沾大恩也、詳見\BookTitle{彌撒祭義}。}
    \item[]{\small 彌撒後誦}
    \begin{quote}\bfseries 聖母經\ {\rm\small 三遍}\quad 申爾福\ {\rm\small 一遍}\end{quote}
\end{enumerate}
\versicle 天主聖母、爲我等祈。\response 以致我等、幸承基利斯督所許洪錫。

請眾同禱。{\cspace}衛我助我者天主。民眾號爾、懇即憐視爾固慈祥仁善。茲因榮光無玷童貞天主之母瑪利亞、及其淨配大聖若瑟、暨宗徒聖伯多祿聖保祿、與諸聖人之代禱。懇祈俯允所求、恩賜罪人悛改、聖教廣揚、脫諸窘難。爲我等主基利斯督。{\cspace}亞孟。

聖彌額爾總領天神。請爾護我於攻魔、衛我於邪神惡計。吾又哀求天主、嚴儆斥之。今魔魁惡鬼、徧散普世、肆害人靈。求爾天上大軍之帥、仗主權能、麾入地獄。{\cspace}亞孟。

\begin{enumerate}
    \item[]{\small 教宗良第十三、詔飭各司祭於彌撒後、虔誦右經。若信友於是時、一心同誦、當獲益良多也。}
    \item[]{\small 教宗庇護第十、又准各司祭、在念前教宗良第十三飭誦經文後、加誦}
\end{enumerate}
耶穌至聖之心、矜憐我等。 {\small 三遍}
\section*{求神愛誦}
吾天主、賜我諸德、尤望賜我聖寵。能以全心全靈、專志愛爾、如允我愛爾。我望之慰、我心之味、我靈之生命、我神體之安所、我明悟之朗耀。求爾除滅我心之偏情、立一大殿於其中、爲爾永居。使愛情之劍透我心、愛情之味養我靈。噫嘻、何時得遂此願。何時盡絕不合爾意之情。何時盡克私意而遵守爾旨。何時純不記憶我身、惟記憶我主。何時盡捐諸物諸事於我心、惟獨存我主。何時愛情之火炙熱我心。何時得享爾之愛情。何時賜我得合爾旨、永遠不離於爾。

至仁至慈皇皇聖三、聖父、聖子、聖神、三位一體。教我治我佑我。聖父、爲爾全能、求爾滿我記含於義善、定志在爾。聖子、聖父之上智。求爾賜我明悟、能通微妙事情。聖神、聖父及聖子所發之愛。爲爾無比之善、求爾賜我聖火。照耀我愛欲、永遠不熄不滅。至聖聖三。賜我愛爾、如天神讚美爾、敬慕爾、方爲真愛。

世人母皇、潔淨之華、福樂之原、童貞之鏡、義德之表。求爾分我希微愛情、能愛爾所愛之子。天上最尊諸天神、以爾愛之火、徧炙宇宙。求爾諸品勿遺棄我、且煉我心之罪、焚之於愛情烈焰。真福聖人聖女、何時能同爾諸品、讚美吾主、愛慕吾主。永遠享其榮福。{\cspace}亞孟。
\section*{彌撒中奉香經文}
\begin{enumerate}
    \item[]{\small 念解罪經後、神父上臺、添香時念。}
\end{enumerate}
爲主榮所焚爇、主即聖之。{\cspace}亞孟。
\begin{enumerate}
    \item[]{\small 移經臺右、添香時、念經如前。}
    \item[]{\small 念}\ 吾主、以此聖血\ {\small 云云後添香時、念}
\end{enumerate}
因聖彌額爾總領天神、侍立香臺右、暨諸聖轉達。望主允降福於此乳香、飴馨享受。爲我等主基利斯督。{\cspace}亞孟。
\begin{enumerate}
    \item[]{\small 鐸德向祭品奉香時、念}
\end{enumerate}
主乎、此乳香、承爾祝聖、升爾臺前。望爾降仁慈於我等。
\begin{enumerate}
    \item[]{\small 鐸德向臺奉香時、念}
\end{enumerate}
我等幸徹爾臺前、如乳香然。吾手之舉、如晚祭然。望主禁持我口、衛我唇吻之門。俾我心無僻言、以辭飾罪。
\begin{enumerate}
    \item[]{\small 鐸德授吊爐於副祭時、念}
\end{enumerate}
望主燃我等以真愛之火、於永仁之焰。{\cspace}亞孟。
\section*{謝聖體經}
\section*{已領聖體祝文}
我謝天主、我感天主。今日臨我胸中、比死後援我升天、厥恩更深、厥功更大。我今永時永日永歲、常憶天主救贖之奇愛。常思聖母敬供之隆情、常如三王來朝之盛禮。周日存想、一生不變。庶得天主、常存我心、常持我口、常保我身。{\cspace}亞孟。
\section*{敬拜經}
吾主耶穌、以爾聖身、降我心中。我今俯伏恭向、敬拜爾恩、欽崇爾德。深謝爾以全能變化。永留聖體大祭於人間、俾我等常荷神佑。即合天朝神聖、天下人民、亦難報於萬一。懇求吾主耶穌、恕我卑賤、補我虧缺。令我時發熱愛、仰慕天上光榮、讚揚我主無對之尊。{\cspace}亞孟。
\section*{愛心經}
吾主耶穌、以爾聖軀寶血、願作我等神糧。耶穌聖心愛我至切。我今以愛還愛、以心體心。以後定志改遷、遵守規誡、補贖所犯罪過、至死不敢負恩負愛。懇求吾主耶穌、賜我熱愛之火、以熱我心。{\cspace}亞孟。
\section*{感謝經}
吾主耶穌、生我養我、救贖於我。今復斂其光榮、以爾尊貴之體、降臨我極污心中、生活我之靈魂。我實感恩無盡、願同天下萬民、稱頌吾主之洪恩。{\cspace}亞孟。
\section*{祈求經}
吾主耶穌基利斯督。懇求耶穌之聖靈佑我、懇求耶穌之聖身救我、懇求耶穌聖傷之血潔淨我、懇求耶穌之苦難聖死、堅我勵我。嗚呼、伏望仁慈之主、垂允我禱、勿棄我、勿拒我、救我於諸仇諸惡 。凡我親戚友朋、亦求主加之聖寵神佑、改邪歸正。迨至死後、俾我等獲睹天主光榮。同諸天神聖人、常享永樂於無窮。{\cspace}亞孟。
\section*{奉獻經}
吾主耶穌、賜我聖身聖血、常存我心、我固時感時謝。但我罪人、無功無德、無可奉獻於主。懇求吾主耶穌、憐我直誠、納我所獻。我願將靈魂肉身、三司五官、全獻於主。奉獻明悟、願明天主奧理。奉獻記含、願想天主經言。奉獻愛欲、願常熱愛於天主。自今而後、獻我目、不觀非理之物。獻我耳、不聽非理之言。獻我口、不出非理之聲。獻我心、不動非理之念。統獻一身、願用恭敬天主。懇求吾主耶穌、加我神力神勇。{\cspace}亞孟。
\section*{領聖體後誦}
至仁至慈天主。我受主無極恩惠、無可稱謝、我重罪多惡、思言行爲、無不得罪、身神污穢。主不但不罰我、更寬裕待我。肫切動我、悔恨前愆、改遷無怠。更忘我罪、容我近主、領主聖體。令我神內、得懷上天下地、無比珍美。主在世時、凡誠心近主、無不取益適願。有罪者被化、改惡遷善。病者獲愈、憂者獲慰、苦者獲安、愚者獲明。今蒙主仁慈、得我罪之赦、我病之愈、我憂之慰、我苦之安、我愚之明。主在我心、爲我心主、求主常居勿棄我。以主聖意、爲我心志、庶恒懷主、須臾不離。善生安死、偕主享主、至於無窮。{\cspace}亞孟。
感謝吾主耶穌、無價珍寶神糧。賜助神力、免陷罪惡、易走天堂之路。
\section*{耶穌聖體禱文\quad{\rm\small 見下瞻禮五禱文}}
\section*{向耶穌苦像誦}
吁、純善至溫之耶穌、我今跪伏爾前、憶昔先知達味代爾言曰、彼輩鑽於手足、數予諸骸。吁、純善耶穌、我將此懸擬於目前。傾心熱衷、懇爾禱爾。乘我摯愛極恫、默睹爾五傷時、即以信望愛活潑之情。我罪之真悔、悛改之堅志、銘刻於我心中。

\begin{enumerate}
    \item[]{\small 右經系教宗庇護第七位參訂。教宗格肋孟德第八位、與本篤第十四位、先後准定。凡告解領主、於主像前悔心虔誦斯經、且誦時有求主爲聖教要務之意者、即得全大赦、亦可讓於煉靈。後教宗庇護第九位重諭、除告解領主悔心虔誦右經外、又須照常加念其他經、按教宗之意求主、方得全赦。}
\end{enumerate}
\begin{quote}\bfseries 天主經、聖母經、聖三光榮頌\quad {\rm\small 各一遍}

領大赦誦\quad{\rm\small 一遍}\par
\end{quote}
